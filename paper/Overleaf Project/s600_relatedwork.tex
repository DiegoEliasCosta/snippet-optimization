% Blank page just to separate Related work at the moment
\newpage
\phantom{blabla}
\newpage

Various studies have investigated the use of SO code on software engineering.


\subsection{Empirical Studies on SO}

% Form query to Usable Code
Yang \cite{Yang:2016_FromQueryToUsableCode} assessed the usability of code snippets.
\begin{itemize}
	\item Usability of code inferred by compiling and running the source-code
	\item Python and JavaScript proved to be the languages with highest rate of usable code
\end{itemize}

% Stack Overflow on GitHub: Any Snippets There?
In \cite{Yang:2017_AnySnippetsThere} authors have investigated the occurrence of SO code in Github projects.
\begin{itemize}
	\item Analyze 909k Python projects containing 290M function definitions from \github.
	\item Analyze 1.9M Python snippets from Stack Overflow
	\item Quantitative analysis: exact duplication between SO and \github exists but is rare, much less than 1\%. Although the percentage is not very large the numbers are in thousands.
	\item Qualitative analysis: Vast majority of duplicated codes are very small.
	\item \diego{We have to further investigate the issues found in this paper. If the percentage of usage is so small, we might run into trouble to find enough data in \github projects.}
\end{itemize}

% Stack Overflow: A code laundering platform?
An \textit{et al.} \cite{An:2017_SOACodeLaunderingPlatform} investigated the unethical reuse of Stack Overflow code in real projects.
\begin{itemize}
	\item Study made on 339 Android applications
	\item Vast majority \diego{(add number here later)} of code snippets used in projects do not report the license from SO (potential code violation).
\end{itemize}

Baltes \textit{et al.}~\cite{Baltes:2017_AttributionRequired} also investigates the occurrences of copied SO snippets in GitHub projects.
\begin{itemize}
	\item Only 23\% of the identified clones of Java snippets have link that attributes it to the original SO post.
    \item 
\end{itemize}

% 
Gabel \textit{et al.} \cite{Gabel:2010_UniquenessOfCode} investigated the property of uniqueness of code in software projects.
\begin{itemize}
	\item General lack of uniqueness in software in levels of granularity equivalent to approximately one to seven lines of code.
	\item Pervasive phenomenon: it happens across projects ans programming languages.
\end{itemize}


\textbf{Dataset.} Ponzanelli \textit{et al.} provided a structured dataset for SO \cite{Ponzanelli:2015:STORM_SODataset} for Java programs.
\begin{itemize}
	\item Model each code as a heterogeneous abstract syntax tree \diego{check this in details}.
	\item Link available at \footnote{https://stormed.inf.usi.ch/}.
	\item \diego{We need to check this dataset if we want to add as a contribution our own manually tuned dataset.}
\end{itemize}

\subsection{Using SO in Software Engineering}

\textbf{Code Recommender.} Ponzanelli's group has worked on several approaches that harness Stack Overflow code snippets into code recommender. 

In \cite{Ponzanelli:2014_MiningSOTurnIDE,Ponzanelli:2014_Prompter}, the authors presented the core idea of mining Stack Overflow to provide Prompter, a IDE plugin that:
\begin{itemize}
	\item Continuously monitor developer's code and try to find the best "next" code, based on SO usage.
    \item Code is ranked based on the similarity of currents developer code (clone code detection).
    \item \diego{Check if authors only mine SO or also GitHub}
\end{itemize}

SeaHawk~\cite{Ponzanelli:2013_SeaHawk} (from the same group) integrates Stack Overflow search with an Eclipse IDE plugin.
\begin{itemize}
	\item Formulates queries automatically from the active context of the IDE
    \item Ranked and interactive list of results
    \item Lets user import code 
    \item YouTube video~\footnote{https://www.youtube.com/watch?v=DkqhiU9FYPI}
    \item \diego{There is no transformation of the code involved (no automated transplantation).}
\end{itemize}

In a recent study~\cite{Ponzanelli:2017_Holistic} addresses the problem of heterogeneity of Stack Overflow posts. In this work the authors propose Libra
\begin{itemize}
	\item Integrate IDE with the Google search in the Web Browser
    \item Classify the posts based on the relevance to the developer's context
    \item Project page \footnote{http://libra.inf.usi.ch/}
	\item YouTube video \footnote{https://www.youtube.com/watch?v=yb69hYwTYyA}
    \item \diego{Our contribution is orthogonal to the contributions proposed by Ponzanelli's group.}
\end{itemize}

\textbf{Program Repair.} In "Mining Stack Overflow for Program Repair"~\footnote{www.cs.sjtu.edu.cn/~zhonghao/paper/minstackoverflow.pdf} (to be published in SANER`18), the authors mined the SO to provide templates for automated program repair:
\begin{itemize}
	\item Template patches are generated based on the Question-Answer code. 
    \item 13 templates were mined from SO
    \item Templates used on Defects4J to fix 23 bugs
\end{itemize}

In~\cite{Nielebock:2017_TowardsAPIAutomaticProgramRepair}, authors introduce the concept of API-Specific Automatic Program Repair. 
\begin{itemize}
	\item \diego{Short paper on ASE 2017}
    \item Discusses the concept of having API-specific repair patches
    \item Points at the challenge of "Extracting API-specific information", which we are tackling. \diego{This can be further used for motivation}.
\end{itemize}

