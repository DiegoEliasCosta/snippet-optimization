% Introduction goes here...

% \begin{itemize}
% 	\item Python has a solid claim to being the fastest-growing major programming language. \footnote{https://insights.stackoverflow.com/survey/2018/\#technology} 
% \end{itemize}


The argumentation should go as follows:


\textbf{Importance of Stack Overflow code snippets.}
\begin{itemize}
	\item Programmers use the web extensively as a source of reusable code.
	\item \neelesh{Stackoverflow is rich bilingual corpus with natural human langauges like English and programming language code. Extracting reusable code snippets could help in Machine Translation (MT) which requires a parallel corpus or bitext.}
	\item Stack Overflow (SO), one of the most accessed web pages, provides an invaluable code-centric knowledge in a form of code snippets. 
	
	\begin{itemize}
		\item Code snippets enriched with relevant textual explanation 
		\item Questions and answers are ranked based on the usefulness to the community. Therefore a high-ranked question might provide snippets reused across multiple programs. 
		\item Answers to the same question seek to provide similar code snippet functionality - snippets with equivalent semantic. A dataset with semantic equivalent code can be applied in a myriad of domains: API recommendation, performance optimization, program synthesis, etc..
		\item In general, automated program repair \neelesh{or MT} could benefit greatly from using Stack Overflow as a source of code snippets \neelesh{and texts.}
	\end{itemize}
	
\end{itemize}	
	

\textbf{Quality of \neelesh{Stackoverflow posts and} code snippets is a Challenge}

\begin{itemize}
	\item Previous work have investigated the suitability of Stack Overflow code snippets on Automated Program Repair \cite{Yang:2016_FromQueryToUsableCode} \neelesh{and Machine Translation \cite{Rahman:2019_CleaningStackOverflowforMT, Yin:2018_LearningMineAlignedCodeNLPairsSO}}.
	
	\begin{itemize}
	    \item \neelesh{People write posts in informal manner like incorrect spellings, abbreviations, inappropriate use of punctuation and gramatical mistakes.}
	    \item \neelesh{Text and code is quite intermixed while explaining the solution.}
	    \item \neelesh{Presence of redundant and unnecessary code which were used to setup a context before providing the actual solution to the question.}
		\item Strong-typed languages such as Java and C\# offer a very low rate of "out-of-the-box" usable snippets.
		\item Python provided the highest rate of parseable (76\%) and compilable (25\%) snippets. 
		\item Around 90\% of the runtime errors were related to the usage of undefined variables, attributes and modules.
		\item Such errors can be mitigated by inferring the context in which the snippets are been applied. For instance, in Python \texttt{np} is often used as an alias for the \texttt{numpy} module, while \texttt{df} is often used as an abbreviation for pandas \texttt{dataframe}.
		
	\end{itemize}		
\end{itemize}		

\textbf{Using API documentation and repository mining to extract reusable code templates from Stack Overflow.}

APIs are a key player in the programming language ecosystems. 
\begin{itemize}
	\item APIs offer a common language for problem solving and are used across programs
	\item \textbf{Intuition.} APIs are related to a specific program domain (logging system, data analysis, data structures). Hence, questions tagged to APIs are constrained to the API domain, reducing the space of search for undefined/incomplete elements of the code.  
	\item Programmers use similar naming conventions when using APIs, by consequence the variable name might be used for data type inference.

\end{itemize}
	
\textbf{Main questions.} Can we combine the API documentation to extract code templates from Stack Overflow? \neelesh{Can these code templates improve the quality of parallel corpus to be built for Machine Translation?}

