Stack Overflow (SO) is one of the most popular QA website used by developers for programming centric discussions, providing invaluable code-centric knowledge enriched with relevant textual explanations. This knowledge has the potential to be used used for various tasks like program synthesis, code summarization, automatic program repair. However, as its current state, SO posts contain a lot of noisy data and scattered irrelevant code used as setup, posing a challenge for its use in automated methods. 

%Before such forums came into existence and even today, API manuals are used as source of truth to find relevant methods to implement a task. These manuals are also enriched with code and exact text information with well-defined objective. This API description could be used to locate relevant code part in SO. 

In this paper, we propose automated methods for identifying relevant code elements from SO posts based on method calls and their documentation from the API. 
As its core, our approach uses the API documentation as a separate source of data for the cleaning procedure. \diego{Add more details here...} \neelesh{We built a small corpus for pandas-python data analysis library from Stack Overflow Dump. After basic cleaning of code and text we separated irrelevant and relevant code by: Heuristics (API calls, Non-Constructor calls, Non-Constructors calls from API only) and NLP text matching (SO post title and API descriptions).}
%method calls and their textual descriptions from our API documentation which are close match to post title text. 
We evaluated our methods on manually annotated dataset, and found that our methods give better text-code pair quality. \neelesh{Our heuristics are also applicable for other languages like Java\cite{Nielebock:2017_TowardsAPIAutomaticProgramRepair}. NLP text matching approach is quite generic in the sense that constructs and their descriptions could be easily found in APIs. Though we showcased our work on python and pandas library but the results could be easily extended to other language and libraries. }
This work paves a way to combine SO information with traditional API information to achieve better methods for various tasks.
%Post-relevant code pair. Further we used methods \cite{Rahman:2019_CleaningStackOverflowforMT} to extract and clean text and paired them relevant code elements identified by our methods. These pairs were evaluated for Machine Translation using the maximum likelihood MT model proposed by \cite{Rahman:2019_CleaningStackOverflowforMT}. We found our methods giving better text-code pair quality. This work paves a way to combine SO information with traditional API information to achieve better methods for various tasks.